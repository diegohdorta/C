\documentclass[consolas,10pt]{article}

\usepackage{jmlr2e}

% Definitions of handy macros can go here

\newcommand{\dataset}{{\cal D}}
\newcommand{\fracpartial}[2]{\frac{\partial #1}{\partial  #2}}

% Heading arguments are {volume}{year}{pages}{submitted}{published}{author-full-names}

\jmlrheading{1}{2019}{1-48}{4/00}{10/00}{Diego Dorta}

% Short headings should be running head and authors last names

\ShortHeadings{Robin Bird\\small}{}
\firstpageno{1}

\begin{document}

\title{ Robin Bird \\
  \large Small and Flexible Machine Learning \\ Library for Embedded Systems}

\author{\name NXP Semiconductors	\email diego.dorta@nxp.com \\
		Diego Dorta \\
       \addr System and Application Enginner}

\editor{}

\maketitle

\begin{abstract}

Robin Bird is a small and flexible machine learning library, written in C\texttt{++}, that
aims to provide simple and extensible implementations of machine learning algorithms
for solving real world problems on embedded systems. Robin Bird provides these 
algorithms as simple command-line programs, Python bindings, and C\texttt{++} classes which
can be used and integrated into low to medium-scale machine learning solutions.\\

\end{abstract}

\begin{keywords}

  Machine Learning, library, C\texttt{++}, 
  
\end{keywords}

\pagebreak

\section{Introduction}

Probabilistic inference has become a core technology in AI,
largely due to developments in graph-theoretic methods for the 
representation and manipulation of complex probability 
distributions~\citep{pearl:88}.  Whether in their guise as 
directed graphs (Bayesian networks) or as undirected graphs (Markov 
random fields), \emph{probabilistic graphical models} have a number 
of virtues as representations of uncertainty and as inference engines.  
Graphical models allow a separation between qualitative, structural
aspects of uncertain knowledge and the quantitative, parametric aspects 
of uncertainty...\\

{\noindent \em Remainder omitted in this sample. See http://www.jmlr.org/papers/ for full paper.}

% Acknowledgements should go at the end, before appendices and references

\acks{We would like to acknowledge support for this project
from the National Science Foundation (NSF grant IIS-9988642)
and the Multidisciplinary Research Program of the Department
of Defense (MURI N00014-00-1-0637). }

% Manual newpage inserted to improve layout of sample file - not
% needed in general before appendices/bibliography.

\newpage

\appendix
\section*{Appendix A.}
\label{app:theorem}

% Note: in this sample, the section number is hard-coded in. Following
% proper LaTeX conventions, it should properly be coded as a reference:

%In this appendix we prove the following theorem from
%Section~\ref{sec:textree-generalization}:

In this appendix we prove the following theorem from
Section~6.2:

\noindent
{\bf Theorem} {\it Let $u,v,w$ be discrete variables such that $v, w$ do
not co-occur with $u$ (i.e., $u\neq0\;\Rightarrow \;v=w=0$ in a given
dataset $\dataset$). Let $N_{v0},N_{w0}$ be the number of data points for
which $v=0, w=0$ respectively, and let $I_{uv},I_{uw}$ be the
respective empirical mutual information values based on the sample
$\dataset$. Then
\[
	N_{v0} \;>\; N_{w0}\;\;\Rightarrow\;\;I_{uv} \;\leq\;I_{uw}
\]
with equality only if $u$ is identically 0.} \hfill\BlackBox

\noindent
{\bf Proof}. We use the notation:
\[
P_v(i) \;=\;\frac{N_v^i}{N},\;\;\;i \neq 0;\;\;\;
P_{v0}\;\equiv\;P_v(0)\; = \;1 - \sum_{i\neq 0}P_v(i).
\]
These values represent the (empirical) probabilities of $v$
taking value $i\neq 0$ and 0 respectively.  Entropies will be denoted
by $H$. We aim to show that $\fracpartial{I_{uv}}{P_{v0}} < 0$....\\

{\noindent \em Remainder omitted in this sample. See http://www.jmlr.org/papers/ for full paper.}


\vskip 0.2in
\bibliography{sample}

\end{document}
